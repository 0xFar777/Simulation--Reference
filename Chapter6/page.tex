\documentclass{book}  
\usepackage{amsmath,lmodern} 
\usepackage{fancyhdr}
\usepackage{fontspec}
\usepackage[UTF8]{ctex}
\usepackage{ragged2e} 
\usepackage{graphicx}
\usepackage{float}
\usepackage{listings}

\setmainfont{Times New Roman}
\renewcommand\tablename{Table}

\begin{document}
\zihao{4}
\centering
节选第六章 \\

\hspace*{\fill} \\
\zihao{5}
\justifying
\noindent
6.13. For the repair model presented in Section 6.7:  \\
(b) Use your program to estimate the mean crash time in the case where $n=4,s=3,F(x)=1-e^{-x}$
,and $G(x)=1-e^{-2x}$  \\
解:这里介绍两种方法:一种基于过程考虑,代码比较长,但算法的效率比较高(就是书上的方法);另一种方法基于状态考虑,
代码只有二十来行,但算法的效率比较低. \\
法一:基于过程,流程图如下(书上P108可参考): \\
\begin{figure}[H]
    \centering
    \includegraphics*[height = 11.2cm, width = 12cm]{gramFile/十三题/13题(法一)流程图.PNG}
    \caption{13题(法一)流程图}
\end{figure}
\noindent
法一全过程解释: \\
系统正常工作所需要的机器数量为$n$,初始备用机器数量为$s$.系统初始化:先模拟生成第一批工作的机器的工作时长,并进行升序排序,最先坏掉
的机器,其坏掉的时间为$errT[1]$,记为$tmin$.进入循环:机器坏掉就要拿去修理,由于同时在修理的机器最多只能有一台,因此如果前一台机器
没有修理好又有新机器坏掉了$(tfix>tmin)$,那么新坏掉的机器只能加入等待队列($r=r+1$);如果机器修好时,还没有新机器坏掉$(tfix \leq
    tmin)$,则(待修理+修理中的机器数)-1,即($r=r-1$).再看此时是否还有机器在等待队列中未修理,如果有就生成修理机器所需的时间$Y$,再
计算出此机器修理好的时间(即$t+Y$).机器坏掉需要立刻从备用池里取出一台替补上,如果备用池里没有机器(意味着坏掉的机器全在以下两种状
态:待修理或正在修,即$r=s$),则系统崩溃退出.如果备用池里还有机器,这时需要模拟生成新加入机器的工作时间$X$,计算其下次坏掉的时间(即
$t+X$),与所有正在工作机器下次坏掉的时间进行升序排序,取最小的记为$tmin$.上述工作完成后,进入下一轮循环,直到程序退出为止.因此法一
(基于过程)的代码如下:
\lstset{language = R}
\begin{lstlisting}
f <- function(n,s) {
  m <- 66666
  result <- vector(length = m) 
  g <- function(n, s) {
    t <- 0
    r <- 0
    errT <- sort(rexp(n, rate = 1))
    tfix <- 1e9
    for(k in 1:10000) {
      tmin <- errT[1]
      if(tfix <= tmin) {
        t <- tfix   
        r <- r-1
        if(r==0) {
          tfix <- 1e9
        }
        if(r > 0) {
          Y <- rexp(1, rate = 2)
          tfix <- t+Y
        }
      } else {  
        t <- tmin 
        if(r==s) {
          return(t)
        } else {
          X <- rexp(1, rate = 1)
          errT <- sort(append(errT[2:length(errT)], t+X))
          if(r==0) {
            Y <- rexp(1, rate = 2)
            tfix <- t+Y
          } 
          r <- r+1
        }
      }
    }
  }
  for(i in 1:m) {
    result[i] <- g(n,s)
  }
  hist(result, freq = F, breaks = 30, 
       xlab = "持续时间", ylab = "频率", 
       main = "持续时间分布图")
  print(mean(result))
  print(sd(result))
}
f(4,3)
\end{lstlisting}
\begin{figure}[H]
    \centering
    \includegraphics*[height = 1cm, width = 2.5cm]{gramFile/十三题/13题(法一)运行截图.PNG}
    \caption{13题(法一)运行截图}
\end{figure}
\begin{figure}[H]
    \centering
    \includegraphics*[height = 6cm, width = 7.5cm]{gramFile/十三题/13题(法一)持续时间分布图.PNG}
    \caption{13题(法一)持续时间分布图}
\end{figure}
\noindent
由输出结果:经过66666次的模拟,发现整个系统持续时间(首次崩溃的发生时间)的均值为1.529677,
标准差为1.063099 \\

\noindent
法二:基于状态,流程图如下: \\
\begin{figure}[H]
    \centering
    \includegraphics*[height = 7.5cm, width = 12cm]{gramFile/十三题/13题(法二)流程图.PNG}
    \caption{13题(法二)流程图}
\end{figure}
\noindent
法二全过程解释: \\
这个思路其实很简单,举个例子就能明白了,就拿本题来说:需要4台机器同时工作,初始时备用池里有3台机器,那么意味
着:第四台机器坏掉的时间不能小于第一台机器修好的时间,第五台机器坏掉的时间不能小于第二台机器修好的时间(第
$i$台机器坏掉的时间不能小于第$i-3$台机器修好的时间),否则意味着没有一台机器处于备用状态.因此:整个算法中
只需要两个核心的状态变量:$errT$和$fixed$,分别记录机器坏掉的时间和机器修好的时间就行了.无需考虑更多的细
节,就无脑算每台机器的$errT$和$fixed$就行了,循坏到直至整个工作系统崩溃.(但不要忘记一点:前三台机器坏掉
是无论如何都可以有机器替补的,因此$fixed$变量的前三个值都计为0) \\
\noindent
因此法二(基于状态)的代码如下:
\lstset{language = R}
\begin{lstlisting}
  f <- function(n, s) {  
    m <- 66666
    result <- vector(length = m)
    g <- function(n, s) {
      fixed <- rep(0, s)
      errT <- sort(rexp(n, rate = 1))
      for(i in 1:6666) {
        fixT <- rexp(1, rate = 2)
        fixed <- append(fixed, 
                 fixT+max(errT[i], fixed[i+s-1]))
        onT <- rexp(1, rate = 1)
        errT <- sort(append(errT, onT+errT[i]))
        if(errT[i] < fixed[i]) {
          return(errT[i])
        }
      }
    }
    for(i in 1:m) {
      result[i] <- g(n,s)
    }
    hist(result, freq = F, breaks = 30, 
         xlab = "持续时间", ylab = "频率", 
         main = "持续时间分布图")
    print(mean(result))
    print(sd(result))
  }
  f(4,3)
\end{lstlisting}
\begin{figure}[H]
    \centering
    \includegraphics*[height = 1cm, width = 2.3cm]{gramFile/十三题/13题(法二)运行截图.PNG}
    \caption{13题(法二)运行截图}
\end{figure}
\begin{figure}[H]
    \centering
    \includegraphics*[height = 6cm, width = 7.5cm]{gramFile/十三题/13题(法二)持续时间分布图.PNG}
    \caption{13题(法二)持续时间分布图}
\end{figure}
\noindent
由输出结果:经过66666次的模拟,发现整个系统持续时间(首次崩溃的发生时间)的均值为1.525008,
标准差为1.058040.两种方法的模拟结果非常相近.\\

\hspace*{\fill} \\

\noindent
6.14. In the model of Section 6.7, suppose that the repair facility consists of two servers, each of
whom takes a random amount of time having distribution G to service a failed machine. Draw a
flow diagram for this system. \\
翻译:在第6.7节的模型中,假设维修设施由两台服务器组成,每台服务器都需要随机分配G的时间来维修故障机器,绘制此系统的流程图 \\
解:同样还是有13题的两种方法:分别基于过程和状态,前者代码量大但算法效率较高;后者代码量小但算法效率较低。除此之外,基于
状态考虑的算法还有个其他的优势:这个优势当只有一台维修设施时基本体现不出来,就是不需要考虑每种情况的细节,这一题很容易对
情况的细节考虑不全,因此如果使用基于状态的算法,可以为我们节约很多思考时间. \\
法一:基于过程,流程图如下: \\
\begin{figure}[H]
    \centering
    \includegraphics*[height = 12.4cm, width = 13.6cm]{gramFile/十四题/14题(法一)流程图.PNG}
    \caption{14题(法一)流程图}
\end{figure}
\noindent
这里是改成了两台设施(分别记为A,B)进行维修,情况多了非常多.一共有以下7个细节需要去注意,所以说用这种基于过程的算法是非常容易漏掉情况的 \\
(1):A和B都闲着没事在发呆(意味着当前系统时间没有坏掉的机器,坏掉的都修好了)  \\
(2):A在修,且A修完后发现B在那闲着(意味着在A修的期间内没有新的坏掉的机器),那么下一台坏掉的机器默认还是给A修  \\
(3):A和B有一方在修,且在其修的期间内,又出现了一台坏的机器,那么这台机器就得A和B中的另外一方去修  \\
(4):A在修B也在修,且在他们修的期间内,又出现了一台坏的机器,那么这台机器只能暂时被放在等待队列  \\
(5):A在修B也在修,如果有机器在等待队列,那么A和B先修好手中的机器的需要把等待队列中的机器拿出来修  \\
(6):A在修B也在修,如果没有机器在等待队列,那么谁先修完谁先休息(在此之前没有新的坏掉的机器)  \\
(7):如果备用池里没有机器,意味着系统在此刻陷入崩溃,循环结束  \\
Tips:这个方法太过于死板,其实是可以通过推演获取到一些高级结论的,这就是后面要说的基于状态的解决方案,此处流程图真全部描述
一遍起码得1500+字篇幅,所以就不描述咯...... \\
\noindent
法一(基于过程)的代码如下:
\lstset{language = R}
\begin{lstlisting}
f <- function(n,s) {
  m <- 16666
  result <- vector(length = m) 
  g <- function(n, s) {
    t <- 0
    r <- 0
    errT <- sort(rexp(n, rate = 1))
    tfixA <- 1e9
    tfixB <- 1e9
    for(k in 1:10000) {
      tmin <- errT[1]
      if((tfixA <= tmin) && (tfixB <= tmin)) {
        t <- min(tfixA, tfixB)  
        if((r==3)||(r==2)) {
          r <- r-2
        } else {
          r <- r-1
        }
        if(r==0) {
          tfixA <- 1e9
          tfixB <- 1e9
        }
        if(r>0) {
          Y <- rexp(1, rate = 2)
          if(tfixA <= tfixB) {
            tfixA <- t+Y
            if(r==1) {
              tfixB <- 1e9
            }
          } else {
            tfixB <- t+Y
            if(r==1) {
              tfixA <- 1e9
            }
          }
        }
      } else if((tfixA <= tmin) || (tfixB <= tmin)){
        t <- min(tfixA, tfixB)
        r <- r-1
        if((r==1)||(r==0)) {
          if(tfixA <= tfixB) {
            tfixA <- 1e9
          } else {
            tfixB <- 1e9
          }
        }
        if(r > 1) {
          Y <- rexp(1, rate = 2)
          if(tfixA <= tfixB) {
            tfixA <- t+Y
          } else {
            tfixB <- t+Y
          }
        }
      } else {  
        t <- tmin  
        if(r==s) {
          return(t)
        } else {
          X <- rexp(1, rate = 1)
          errT <- sort(append(errT[2:length(errT)], t+X))
          Y <- rexp(1, rate = 2)
          if(r==0) {
            tfixA <- t+Y
          } 
          if(r==1) {
            if(tfixA==1e9) {
              tfixA <- t+Y
            } 
            if(tfixB==1e9) {
              tfixB <- t+Y
            }
          } 
          r <- r+1
        }
      }
    }
  }
  for(i in 1:m) {
    result[i] <- g(n,s)
  }
  hist(result, freq = F, breaks = 30, 
       xlab = "持续时间", ylab = "频率", 
       main = "持续时间分布图")
  print(mean(result))
  print(sd(result))
}
f(4,3)
\end{lstlisting}
\begin{figure}[H]
    \centering
    \includegraphics*[height = 1cm, width = 2.4cm]{gramFile/十四题/14题(法一)运行截图.PNG}
    \caption{14题(法一)运行截图}
\end{figure}
\begin{figure}[H]
    \centering
    \includegraphics*[height = 6cm, width = 7.5cm]{gramFile/十四题/14题(法一)持续时间分布图.PNG}
    \caption{14题(法一)持续时间分布图}
\end{figure}
\noindent
由输出结果:经过16666次的模拟,发现当系统有两台维修设施时,整个系统持续时间(首次崩溃的发生时间)的均值
提升到了2.117459,标准差为1.695110 \\

\noindent
法二:基于状态,流程图如下: \\
\begin{figure}[H]
    \centering
    \includegraphics*[height = 9.5cm, width = 12cm]{gramFile/十四题/14题(法二)流程图.PNG}
    \caption{14题(法二)流程图}
\end{figure}
\noindent
法二全过程解释: \\
这个14题相较于上面的13题,其实就只改变了一层逻辑:"重新计算机器开始维修的时间".所以完全可以继承13题法二
的解决方案,在此之上改变这一层逻辑就行了.具体该如何计算呢?首先机器要维修的前提是它必须是坏掉,因此需要判
断机器啥时候坏,坏掉时的系统时间记为errT[i].其次,一台机器想要被修,必须确保A和B两方维修设施有一方率先完
成了手里正在维修的机器,A和B手里机器的情况有:其中一方手里必定是上一台坏掉的机器;另一方手里是除开上一台
坏掉的机器外,其他所有已坏掉的机器都有可能. \\
举个例子:假如第11台坏掉的机器准备维修,那么A和B其中一方的手里必定是第10台坏掉的机器(这个其实不难理解:假
如两方手里正在修的都不是第10台机器,比方说A和B分别维修的是第8台和第9台机器吧,那么先修好的那一方必定要拿
第10个坏掉的机器来修,而不是拿第11台坏掉的机器过来修,即先坏的必定比后坏的更早修,因此两方手里正在维修的必
定有第10台坏掉的机器);而另一方是第1到9台坏掉的机器都有可能(这里很容易会误会为:另一方只能是第9台坏掉的机
器,其实不是的,你遗漏了以下这种极端情况:第1台机器坏掉了A去维修,需要七七四十九个小时才能修好,而第2到第10
台坏掉的机器可能分别都只需要6小时,意味着B修了第2到第10台坏掉的机器,而在B正在修第10台坏掉的机器时,A终于
把第1台坏掉的机器修好了.你说第11台坏掉的机器应该谁去修?明显是A去修啊),因此第i台坏掉的机器还跟以下这两个
变量扯上关系:第i-1台机器修理完成的时间和第1到第(i-2)台坏掉的机器中最晚修理好的时间.两者取个最小值,再与
errT[i]取个最大值,就是第i台坏掉的机器开始维修的时间,最后再加上这台机器的维修所需时间Y,即为最终修理好的
时间. \\
因此不难发现,还是只需要两个状态变量:$errT$和$fixed$,分别记录每台机器坏掉的时间和修好的时间就行了(注意,
$fixed$变量初始化时前$s$个值都为0).无需考虑更多的细节,就无脑算每台机器的$errT$和$fixed$就行了,无非就
是多了个临时变量用来记录第1到(i-2)个坏掉的机器谁最晚修理完成而已  \\
其实还有一个需要注意的点:何时退出循环?这跟13题只有一台维修设备时还不太一样,不过举例例子肯定能懂了(还是
13题(b)的例子):第5台机器坏掉了,$fixed$数组中前7个数据大于第5台机器坏掉的时间的不能大于等于3个(第$i$台
机器坏掉了,$fixed$数组中前$i+s-1$个数据大于第$i$台机器坏掉的时间的不能大于等于$s$个)
\lstset{language = R}
\begin{lstlisting}
f <- function(n, s) {  
  m <- 16666
  result <- vector(length = m)
  g <- function(n, s) {
    fixed <- rep(0, s)
    errT <- sort(rexp(n, rate = 1))
    lastMAX <- 0
    for(i in 1:666) {
      if(i>2){
        lastMAX <- max(fixed[(s+1):(length(fixed)-1)])
      }
      fixT <- rexp(1, rate = 2)
      fixed <- append(fixed, fixT+max(errT[i],
                      min(fixed[i+s-1], lastMAX)))
      onT <- rexp(1, rate = 1)
      errT <- sort(append(errT, onT+errT[i]))
      if(sum(fixed[1:i+s-1]>errT[i]) >= s) {
        return(errT[i])
      }
    }
  }
  for(i in 1:m) {
    result[i] <- g(n,s)
  }
  hist(result, freq = F, breaks = 20, 
       xlab = "持续时间", ylab = "频率", 
       main = "持续时间分布图")
  print(mean(result))
  print(sd(result))
}
f(4,3)
\end{lstlisting}
\begin{figure}[H]
    \centering
    \includegraphics*[height = 1cm, width = 2.3cm]{gramFile/十四题/14题(法二)运行截图.PNG}
    \caption{14题(法二)运行截图}
\end{figure}
\begin{figure}[H]
    \centering
    \includegraphics*[height = 5.7cm, width = 7.2cm]{gramFile/十四题/14题(法二)持续时间分布图.PNG}
    \caption{14题(法二)持续时间分布图}
\end{figure}
\noindent
由输出结果:经过16666次的模拟,发现当系统有两台维修设施时,整个系统持续时间(首次崩溃的发生时间)的均值
提升到了2.111064,标准差为1.703650.两种方法的模拟结果近似. \\

\noindent
附图:14题(法一)单次运行过程变量输出值: \\
从上到下,首先三个一行的,从左到右依次是:A维修完成手上坏掉的机器时的系统时间,B维修完成手上坏掉的机器时的系统时间,
系统中正在维修和等待维修的机器数量之和;有14个数据的那串是errT变量
\begin{figure}[H]
    \centering
    \includegraphics*[height = 5cm, width = 8.8cm]{gramFile/十四题/14题(法一)单次运行过程变量输出结果.PNG}
    \caption{14题(法一)单次运行过程变量输出结果}
\end{figure}
 
\noindent
附图:14题(法二)单次运行过程变量输出值: \\
从上到下:首先一行一个数据的,是每个坏掉的机器需要修理的时间;有13个数据的那串是errT变量;有12个数据的那
串是fixed变量  \\
\begin{figure}[H]
    \centering
    \includegraphics*[height = 4cm, width = 8.8cm]{gramFile/十四题/14题(法二)单次运行过程变量输出结果.PNG}
    \caption{14题(法二)单次运行过程变量输出结果}
\end{figure}

\end{document}
