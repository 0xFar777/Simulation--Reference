\documentclass{book}  
\usepackage{amsmath,lmodern} 
\usepackage{fancyhdr}
\usepackage{fontspec}
\usepackage[UTF8]{ctex}
\usepackage{ragged2e}
\usepackage{graphicx}
\usepackage{float}
\usepackage{listings}

\setmainfont{Times New Roman}
\renewcommand\tablename{Table}

\begin{document}
\zihao{4} 
\centering
节选第三章 \\

\hspace*{\fill} \\

\zihao{5}
\justifying
\noindent
3.3.$\int_{0}^{1}exp\left\{e^{x}\right\}dx$ \\
\noindent
解:代码如下
\lstset{language = R}
\begin{lstlisting}
    #法一:随机模拟
    r <- runif(10000, min = 0, max = 1)
    mean(exp(exp(r)))
    #法二:直接用积分函数求解
    f <- function(x){exp(exp(x))}
    integrate(f, lower = 0, upper = 1)
\end{lstlisting}
第3题运行截图
\begin{figure}[H]
    \centering
    \includegraphics*[height = 2.8cm, width = 6cm]{gramFile/第三题运行截图.PNG}
    \caption{第三题R代码}
\end{figure}

\hspace*{\fill} \\

\noindent
3.7.$\int_{-\infty}^{\infty}e^{-x^{2}}dx$ \\
这一题用随机模拟需要注意一点:由于$R$语言中的$runif$函数不能输入$-Inf$(负无穷)和$Inf$
(正无穷),因此可以先两边同时取个对数:$\ln{y} = -x^2$,再转化为:
$\sqrt{\ln{\frac{1}{y}}} = x^2$,因此原积分转化为:
$2\int_{0}^{1}\sqrt{\ln\frac{1}{y}}dy$ \\
解:代码如下
\lstset{language = R}
\begin{lstlisting}
    #法一:随机模拟
    r <- runif(10000, min = 0, max = 1)
    mean(2 * sqrt(log(1/r)))
    #法二:直接用积分函数求解
    f <- function(x){exp(-(x^2))}
    integrate(f, lower = -Inf, upper = Inf)
    #法三:对数转化再用积分函数求解
    f <- function(x){2 * sqrt(log(1/x))}
    integrate(f, lower = 0, upper = 1)
\end{lstlisting}
第7题运行截图
\begin{figure}[H]
    \centering
    \includegraphics*[height = 3.9cm, width = 7.2cm]{gramFile/第七题运行截图.PNG}
    \caption{第七题R代码}
\end{figure}

\hspace*{\fill} \\

\noindent
3.9.$\int_{0}^{\infty}\int_{0}^{x}e^{-(x+y)}dydx$   \\
解:原式 = $\int_{0}^{\infty}e^{-x}\int_{0}^{x}e^{-y}dydx$ = $\int_{0}^{\infty}e^{-x}(1-e^{-x})dx$  \\
\indent
令$y = e^{-x}$,则$\int_{1}^{0}(y-1)dy$ = $\int_{0}^{1}(1-y)dy$
\lstset{language = R}
\begin{lstlisting}
    #法一:平均值模拟
    r <- runif(10000, min = 0, max = 1)
    mean(1-r)
    #法二:蒙特卡洛模拟
    n <- 10000
    x<-runif(n, min = 0, max = 1)
    y<-runif(n, min = 0, max = 1)
    x1<--log(x)
    y1<--log(y)
    z<-sum((exp(-x1-y1)/(x*y))[x1>y1])/n
    z
\end{lstlisting}
第9题运行截图
\begin{figure}[H]
    \centering
    \includegraphics*[height = 4.9cm, width = 7.2cm]{gramFile/第九题运行截图.PNG}
    \caption{第九题R代码}
\end{figure}

\hspace*{\fill} \\

\noindent
3.11.Let U be uniform on (0,1). Use simulation to approximate the following:\\
(a) $Corr(U,\sqrt{1-U^2})$ \\
(b) $Corr(U^2,\sqrt{1-U^2})$ \\
解:思路是运用公式$Corr(x,y) = \frac{Cov(x,y)}{\sigma_{x}\sigma_{y}} = \frac{E(xy)-E(x)E(y)}{\sigma_{x}\sigma_{y}}$  \\
(a)题代码:
\lstset{language = R}
\begin{lstlisting}
    r <- runif(10000, min = 0, max = 1)
    EXY <- mean(r * sqrt(1-r^2))
    EXEY <- mean(r) * mean(sqrt(1-r^2))
    sigma <- sd(r) * sd(sqrt(1-r^2))
    corr <- (EXY - EXEY) / sigma
    corr 
\end{lstlisting}
(b)题代码:
\lstset{language = R}
\begin{lstlisting}
    r <- runif(10000, min = 0, max = 1)
    EXY <- mean(r^2 * sqrt(1-r^2))
    EXEY <- mean(r^2) * mean(sqrt(1-r^2))  
    sigma <- sd(r^2) * sd(sqrt(1-r^2))
    corr <- (EXY - EXEY) / sigma
    corr 
\end{lstlisting}
第11题运行截图
\begin{figure}[H]
    \centering
    \includegraphics*[height = 4.8cm, width = 7.5cm]{gramFile/第十一题运行截图.PNG}
    \caption{第十一题R代码}
\end{figure}

\hspace*{\fill} \\

\noindent
3.12.For uniform(0,1) random variables $U1$, $U2$, ... define
$$
    N = Minimum
    \left\{
    \begin{array}{l}
        n:\sum\limits_{i=1}\limits^{n} U_{i} > 1
    \end{array}
    \right\}
$$
That is, N is equal to the number of random numbers that must be summed to exceed 1. \\
(a) Estimate E[N] by generating 100 values of N. \\
(b) Estimate E[N] by generating 1000 values of N. \\
(c) Estimate E[N] by generating 10000 values of N. \\
(d) What do you think is the values of E[N]? \\
解:(a)-(c)题代码如下
\lstset{language = R}
\begin{lstlisting}
    f <- function(n){
        result <- vector(length = n)
        for(i in 1:n) {
            r <- runif(10000, min = 0, max = 1)
            sum = 0
            for(j in 1:length(r)) {
                sum = sum + r[j]
                if(sum > 1) {
                    result[i] = j
                    break
                }
            }
        }
        print(mean(result))
    }
    f(100)
    f(1000)
    f(10000)
\end{lstlisting}
第十二题运行截图
\begin{figure}[H]
    \centering
    \includegraphics*[height = 2cm, width = 2cm]{gramFile/第十二题运行截图.PNG}
    \caption{第十二题R代码}
\end{figure}
\noindent
(d)题:结合(a)-(c)题结果,可以猜测E[N]的值为自然底数$e$

\hspace*{\fill} \\

\noindent
3.13. Let $U_{i}$, $i \geq 1$, be random numbers.Define $N$ by
$$
    N = Minimum
    \left\{
    \begin{array}{l}
        n:\prod_{i=1}^{0}U_{i} > 1
    \end{array}
    \right\}
$$
where $\prod_{i=1}^{0}U_{i} \equiv 1$. \\
(a) Find E[N] by simulation \\
(b) Find $P\left\{N=i\right\}$, for i = 0,1,2,3,4,5,6, by simulation. \\
解:(a)题代码如下
\lstset{language = R}
\begin{lstlisting}
    n <- 10000
    result <- vector(length = n)
    for(i in 1:n) {
        r <- runif(10000, min = 0, max = 1)
        accumulate = 1
        for(j in 1:length(r)) {
            accumulate = accumulate * r[j]
            if(accumulate < exp(-3)) {
                result[i] = j-1
                 break
            }
        }
    }
    mean(result)
\end{lstlisting}
第十三题(a)运行截图
\begin{figure}[H]
    \centering
    \includegraphics*[height = 6cm, width = 8cm]{gramFile/第十三题(a)运行截图.PNG}
    \caption{第十三题(a)R代码}
\end{figure}
\noindent
(b)题代码如下:
\lstset{language = R}
\begin{lstlisting}
    n <- 10000
    count <- vector(length = 7)
    for(i in 1:n) {
        r <- runif(10000, min = 0, max = 1)
        accumulate = 1
        for(j in 1:length(r)) {
            accumulate = accumulate * r[j]
            if(accumulate < exp(-3)) {
                if(j-1 <= 6) {
                    count[j] = count[j] + 1
                }
                 break
            }
        }
    }
    for(k in 1:length(count)) {
        print(count[k]/n)
    }
\end{lstlisting}
第十三题(b)运行截图
\begin{figure}[H]
    \centering
    \includegraphics*[height = 9cm, width = 8cm]{gramFile/第十三题(b)运行截图.PNG}
    \caption{第十三题(b)R代码}
\end{figure}
\noindent
所以:
\begin{center}
    \begin{tabular}{|c|c|}
        \hline
        $P\left\{N=0\right\} = 0.0504$ & $P\left\{N=1\right\} = 0.1556$ \\
        \hline
        $P\left\{N=2\right\} = 0.2199$ & $P\left\{N=3\right\} = 0.2166$ \\
        \hline
        $P\left\{N=4\right\} = 0.1736$ & $P\left\{N=5\right\} = 0.0993$ \\
        \hline
        $P\left\{N=6\right\} = 0.0525$ & -----                          \\
        \hline
    \end{tabular}
\end{center}

\hspace*{\fill} \\

\noindent
3.14. With $x_{1} = 23$, $x_{2} = 66$, and \\
$$
    x_{n} = 3x_{n-1} + 5x_{n-2} \ \ \ \ \ mod(100), \ \ \ n \geq 3
$$
We will call the sequence $u_{n} = x_{n}/100, \ n \geq 1$,
the text's random number sequence. Find its first 14 values. \\
解:代码如下
\lstset{language = R}
\begin{lstlisting}
    result <- vector(length = 14)
    x <- vector(length = 14)
    result[1] = 23
    result[2] = 66
    x[1] = 23 %% 100
    x[2] = 66 %% 100
    for(i in 3:14) {
        x[i] = 3 * result[i-1] + 5 * result[i-2]	
        result[i] = x[i] %% 100
    }
    print(round(result/100,2))
\end{lstlisting}
第十四题运行截图
\begin{figure}[H]
    \centering
    \includegraphics*[height = 4.2cm, width = 12cm]{gramFile/第十四题运行截图.PNG}
    \caption{第十四题R代码} 
\end{figure}   
 
\end{document}