% !TeX program = xelatex
\documentclass{book}  
\usepackage{amsmath,lmodern} 
\usepackage{fancyhdr}
\usepackage{fontspec}
\usepackage[UTF8]{ctex}
\usepackage{ragged2e}
\usepackage{graphicx}
\usepackage{float}
\usepackage{listings}

\setmainfont{Times New Roman}
\renewcommand\tablename{Table}


\begin{document}

\zihao{4}
\centering
节选第一、二章 \\

\zihao{5}
\justifying
\noindent
1.1. The following data yield the arrival times and service times that
each customer will require,for the first 13 customer will require,
for the first 13 customers at a single server system. Upon arrival,
a customer either enters service if the server is free or joins
the waiting line. When the server completes work on a customer,
the next one in line enters service. \\
\noindent
Arrival Times: 12 31 63 95 99 154 198 221 304 346 411 455 537 \\
Service Time: 40 32 55 48 18 50 47 18 28 54 40 72 12 \\
\indent
(a) Determine the departure times of these 13 customers  \\
\indent
(b) Repeat(a) when there are two serves and a customer can be served by either one \\
\indent
(c) Repeat(a) under the new assumption that when the server completes a service,
the next customer to enter service in the one who has been waiting the least time \\
\noindent
翻译:
以下给出了13个客户需要的到达时间和服务时间,到达后,客户要么在服务器空闲时进入服务,要么加入等
待队列。当服务器完成对客户的工作时,下一个排队的客户进入服务 \\
到达时间: 12 31 63 95 99 154 198 221 304 346 411 455 537 \\
服务时间: 40 32 55 48 18 50 47 18 28 54 40 72 12 \\
\indent
(a)确定这13位客户的离开时间 \\
\indent
(b)当有两个服务并且客户可以由其中一个服务时 \\
\indent
(c)重复(a),使得当服务器完成服务时,下一个进入服务的客户是等待时间最少的客户 \\
\noindent
Answers: \\
(a)题解析:
\begin{tabular}{c c c}
    第$n$位 & 服务是否存在空闲      & 离开时间                   \\
    1     & $0 < 12$,是    & $0+12+40=52$           \\
    2     & $52 > 31$,否   & $52+0+32=84$           \\
    3     & $84 > 63$,否   & $84+0+55=139$          \\
    4     & $139 > 95$,否  & $139+0+48=187$         \\
    5     & $187 > 99$,否  & $187+0+18=205$         \\
    6     & $205 > 154$,否 & $205+0+50=255$         \\
    7     & $255 > 198$,否 & $255+0+47=302$         \\
    8     & $302 > 221$,否 & $302+0+18=320$         \\
    9     & $320 > 304$,否 & $320+0+28=348$         \\
    10    & $348 > 346$,否 & $348+0+54=402$         \\
    11    & $402 < 411$,是 & $402+(411-402)+40=451$ \\
    12    & $451 < 455$,是 & $451+(455-451)+72=527$ \\
    13    & $527 < 537$,是 & $527+(537-527)+12=549$ \\
\end{tabular}

\hspace*{\fill} \\
(b)题解析:\\
\begin{tabular}{c c c c}
    第$n$位 & 服务器A是否空闲      & 服务器B是否空闲      & 离开时间                 \\
    1     & $0 < 12$,是    & ---           & $0+12+40=52$         \\
    2     & $52 > 31$,否   & $0 < 31$,是    & $31+0+32=63$         \\
    3     & $52 < 63$,是   & ---           & $63+0+55=118$        \\
    4     & $118 > 95$,否  & $63 < 95$,是   & $95+0+48=143$        \\
    5     & $118 > 99$,否  & $143 > 99$,否  & $99+(118-99)+18=136$ \\
    6     & $136 < 154$,是 & ---           & $154+0+50=204$       \\
    7     & $204 > 198$,否 & $143 < 198$,是 & $198+0+47=245$       \\
    8     & $204 < 221$,是 & ---           & $221+0+18=239$       \\
    9     & $239 < 304$,是 & ---           & $304+0+28=332$       \\
    10    & $332 < 346$,是 & ---           & $346+0+54=400$       \\
    11    & $400 < 411$,是 & ---           & $411+0+40=451$       \\
    12    & $451 < 455$,是 & ---           & $455+0+72=527$       \\
    13    & $527 < 537$,是 & ---           & $537+0+12=549$       \\
\end{tabular}

\hspace*{\fill} \\
(c)题解析:代码见P3图1,运行截图见P4图2\\
\lstset{language = C++}
\begin{lstlisting}
    #include<bits/stdc++.h> 
    using namespace std;
    
    const int N = 13;
    deque<int> arrive;
    deque<int> service;
    deque<int> wait;
    
    int main() {
        int tmp = 0;
        int n = N, m = N;
        while(n--) {
            cin >> tmp;
            arrive.push_back(tmp);
        }
        while(m--) {
            cin >> tmp;
            service.push_back(tmp);
        }
        arrive.push_back(1e9);
        service.push_back(1e9);
        int finishTime = arrive[0];
        int i = 0;
        while(i < N){
            while(finishTime >= arrive[i]){
                wait.push_back(service[i]);
                i++;	
            } 
            while(finishTime < arrive[i]) {
                if(!wait.empty()) {
                    finishTime += wait[wait.size() - 1];
                    wait.pop_back();
                } else {
                    finishTime += service[i];
                }
                cout << "离开时间是" << finishTime << endl;
                if(finishTime < arrive[i] && wait.size() == 0) {
                    finishTime = arrive[i];
                }
            } 
        }
    }
\end{lstlisting}

\begin{figure}[H]
    \centering
    \includegraphics*[]{gramFile/1(c).png}
    \caption{第一题(c)运行截图}
\end{figure}

\hspace*{\fill} \\

\noindent
1.2. Consider a service station where customers arrive and are served in their orders
of arrival. Let $A_{n}$, $S_{n}$ and $D_{n}$ denote, respectively, the arrival time,
the service time, and the departure time of customer $n$. Suppose there is a single
server and the system is initially empty of customers. \\
\indent
(a) With $D_{0} = 0$, argue that for $n > 0$
$$
    D_{n} - S_{n} = Maximum\left\{A_{n}, D_{n-1}\right\} \\
$$
\indent
(b) Determine the corresponding recursion formula when there are two servers. \\
\indent
(c) Determine the corresponding recursion formula when there are $k$ servers. \\
\indent
(d) Write a computer program to determine the departure times as a function of the arrival and
service times and use it to check your answers in parts (a) and (b) of Exercise 1. \\
\noindent
翻译:
一个服务站按照顾客的到达顺序提供服务。设$A_{n}$、$S_{n}$和$D_{n}$分别表示客户$n$的到达时间、
服务时间和离开时间。假设只有一台服务器,系统最初没有客户 \\
\indent
(a)若$D_{0} = 0$,讨论当$n > 0$时
$$
    D_{n} - S_{n} = Maximum\left\{A_{n}, D_{n-1}\right\} \\
$$
\indent
(b)当有两台服务器时,确定相应的递归公式 \\
\indent
(c)当有k台服务器时,确定相应的递归公式 \\
\indent
(d)编写一个计算机程序,根据到达和服务时间推算顾客的离开时间,并用它检查练习1(a)和(b)部分的答案. \\
\noindent
Answer: \\
(a)题解析: \\
$D_{n}-S_{n}$的实际含义是服务器开始给顾客提供服务的时间,而这又取决于两点:顾客何时
来以及来了是否需要等侯,因此完全取决于$D_{n-1}$(上一个顾客离开的时间)和$A_{n}$(顾
客到达时间)哪个大($D_{n-1} > A_{n}$说明顾客需要等,服务开始的时间就等于$D_{n-1}$;
$D_{n-1} \leq A_{n}$说明顾客不需要等,服务开始的时间就等于$A_{n}$) \\
(b)题解析: \\
答案: 本题用文字解释很简单,$D_{n} - S_{n}$(即顾客开始办理业务的时间),取决于两点:
一个是$A_{n}$(即顾客到达的时间),人没到肯定办不了业务;另一个是取决于两台服务器正在
服务的客户当中最先办理好业务的那位所离开的时间,最后再与$A_{n}$取个最大值,就是用户开
始办理业务的时间。需要注意的是:假如客户$k$需要等待,意味着两台服务器均有人,但是正在
办理业务的两个人不一定是$k-1$和$k-2$,可能发生以下这种情况:就是$A$和$B$两台服务器,
其中的$A$服务器正在为顾客$k-3$服务,但是这人很麻烦,需要办理十小时的业务,但是后面进
来的$k-2$和$k-1$顾客,办理业务都只需要一分钟,等$k-2$顾客在$B$服务器办理完业务的时
候,$k-1$只能去$B$服务器,因为$A$服务器还在为$k-3$客户服务.甚至,等客户$k-1$办理完
业务的时候,顾客$k-3$依旧未完成业务办理,因此:顾客$k$必须等第$k-1$位顾客在服务器$B$
完成业务办理后,才能去$B$服务器接受业务办理.因此,$D_{n}-S_{n}$并不是取决于
$Maximum\left\{D_{n-1},D_{n-2}\right\}$,而是取决于两台服务器正在服务的客户当中最先
办理好业务的那位所离开的时间,正在接受服务的两个人,不一定是$k-1$和$k-2$,也不一定是$k-3$
和$k-1$,有很多种可能,甚至可能出现以下这种极端情况:第一个顾客进来后,一直办理服务直到
当天服务器下班才结束,这样,对于其他后面进来办理业务的顾客来说,服务站就退化为了只有一台
服务器,因此必须等到第$k-1$位顾客离开第$k$位才能开始办理业务.但可以确认的是:第$k$位
顾客想要开始办理业务的前置条件是:第$k-1$位顾客已经在接受业务办理了,意思就是第$k$位顾
客不可能比第$k-1$位顾客更先开始办理业务. \\
因此递推公式为:
$$
    D_{n} - S_{n} = Max\left\{A_{n},min(D_{n-1},D_{n-r})\right\} ((r \in N^{+})
    \bigcap (r \in [2,3,4...n-1]))
$$
(c)题解析: \\
思路和(b)一模一样,不再赘述,有$k$台服务器意味着第$n$个顾客开始办理业务的
时间取决于正在办理业务的$k$个顾客最先离开的那位,当然还取决于第$k$个顾客
自己到达服务站的时间,最后两者再取个最大值,就是$D_{n} - S_{n}$的值. \\
因此递推公式为:
$$
    D_{n} - S_{n} = Max\left\{A_{n},min(D_{n-1},D_{n-r_{1}},D_{n-r_{2}},
    ..., D_{n-r_{k-1}})\right\}
$$
且需要满足: \\
$$
    ((r_{1},r_{2},...,r_{k-1} \in N^{+})\bigcap (r_{1},r_{2},...,
    r_{k-1} \in [2,3,4...n-1])
$$
$$
    ((\forall i,j \in (N^{+} \bigcap  [2,3,4...n-1])) \bigcap (i \neq j)), r_{i} \neq r_{j}
$$
(d)题解析: \\
\lstset{language = C++}
\begin{lstlisting}
    #include<bits/stdc++.h>
    using namespace std;
    
    const int M = 100005;
    int K, N;
    deque<int> arrive;
    deque<int> service;
    deque<int> wait;
    deque<int> isService;
    int finishTime[M];
    
    int main() {
        int tmp = 0;
        cin >> K >> N;
        int n = N, m = N;
        while(n--) {
            cin >> tmp;
            arrive.push_back(tmp);
        }
        while(m--) {
            cin >> tmp;
            service.push_back(tmp);
        }
        int use = 0;
        int NowTime = 0;
        for(int i = 0; i < N; i++) {
            NowTime = arrive[i];
            sort(isService.begin(), isService.end());
            while(NowTime > isService[0] && isService.size()) {
                isService.pop_front();
                use--;
            }
            while(wait.size() && use < K) {
                isService.push_back(wait[0]);
                wait.pop_front();
                use++;
            }
            sort(isService.begin(), isService.end());
            int f = 0;
            if(use < K) {
                f = arrive[i] + service[i];
                isService.push_back(f);
                use++;
            } else {
                if(wait.size() >= K) {
                    f = service[i] + wait[wait.size() - K];
                } else {
                    f = service[i] + isService[wait.size()];
                }
                wait.push_back(f);
            }
            finishTime[i] = f;
        }
        for(int v = 1; v <= N; v++) {
            cout << "第" << v << "位顾客的离开时间是" 
            cout << finishTime[v-1] << endl; 
        }
    } 
\end{lstlisting}

第一题(a)运行截图
\begin{figure}[H]
    \centering
    \includegraphics*[]{gramFile/2(d)--1(a).png}
    \caption{第一题(a)C++代码}
\end{figure}

第一题(b)运行截图
\begin{figure}[H]
    \centering
    \includegraphics*[]{gramFile/2(d)--1(b).png}
    \caption{第一题(b)C++代码}
\end{figure}

\hspace*{\fill} \\

\noindent
2.29. Persons A,B,and C are waiting at a bank having two tellers
when it opens in the morning.Persons A and B each go to a teller
and C waits in line. If the time it takes to serve a customer
is an exponential random variables with parameter λ, what is
the probability that C is the last to leave the bank? \\
\noindent
翻译:
早上银行开门时,A、B和C三个人正在一家有两个出纳员的银行等候。
A和B各自去出纳员那里,C排队等候。如果为客户服务所需的时间是一个参数
为$\lambda$的指数随机变量,那么C最后离开银行的概率是多少? \\
\noindent
Answer:
\begin{align*}
    P(C_{last})
     & = 2P(A_{before B} \&\& B_{before C})           \\
     & = 2P(B_{before C}|A_{before B})P(A_{before B})
\end{align*}
\noindent
由于指数分布的无记忆性,A在B之前离开后,B和C谁先走谁后走的概率都是$\frac{1}{2}$,
因此:$P(B_{before C}|A_{before B}) = \frac{1}{2}$,所以,$P(C_{last})$ =
$2*\frac{1}{2}*\frac{1}{2}$ \\

\noindent
2.32. For a Poisson process with rate λ,find $P\left\{N(s) = k|
    N(t) = n\right\}$ when $s < t$.\\
\noindent
翻译:对于速率为$\lambda$的泊松过程,当$s < t$时,求出$P\left\{N(s)
    =k|N(t)=n\right\}$.\\
\noindent
Answer:
\begin{align*}
    P\left\{N(s)=k|N(t)=n\right\}
     & = \frac{P(N(s)=k,N(t)=n)}{P(N(t)=n)}                 \\
     & = \frac{P(N(s)=k,N(t)-N(s)=n-k)}{P(N(t)=n)}          \\
     & = \frac{P(N(s)=k)P(N(t-s)=n-k)}{P(N(t)=n)}           \\
     & = \frac{\frac{(\lambda s)^{k}}{k!} e^{-\lambda
                s} \frac{(\lambda (t-s))^{n-k}}{(n-k)!} e^{-\lambda
    (t-s)}}{\frac{(\lambda t)^{n}}{n!} e^{-\lambda t}}      \\
     & = \frac{s^{k}(t-s)^{n-k}}{t^{n}} \frac{n!}{k!(n-k)!}
\end{align*}

\noindent
2.33. Repeat Exercise 32 for $s > t$.\\
\noindent
翻译:当$s > t$时,32题的结果如何?\\
\noindent
Answer:
\begin{align*}
    P\left\{N(s)=k|N(t)=n\right\}
     & = \frac{P(N(s)=k,N(t)=n)}{P(N(t)=n)}                      \\
     & = \frac{P(N(s)-N(t)=k-n,N(t)=n)}{P(N(t)=n)}               \\
     & = P(N(s-t))=k-n                                           \\
     & = \frac{(\lambda (s-t))^{k-n}}{(k-n)!} e^{-\lambda (s-t)} \\
\end{align*}

\noindent
2.36. If $X$ and $Y$ are independent and identically distributed
exponential random variables, show that the conditional
distributions of $X$, give that $X+Y=t$, is the uniform
distribution on $(0,t)$.\\
\noindent
翻译:如果$X$和$Y$是独立且同分布的指数随机变量,在给定$X+Y=t$的前提下,
证明随机变量$X$服从$(0,t)$上的均匀分布.\\
\noindent
Answer: \\
由于X和Y均服从Exp($\lambda$),且指数分布同时也是$\alpha = 1$
的伽马分布,由伽马分布的可加性,可知X+Y服从Ga(2,$\lambda$)。所
以对于$x\in[0,t]$,有:
$$
    P\left\{X=x|X+Y=t\right\} = \frac{P(X=x,Y=t-x)}{P(X+Y=t)}
    = \frac{\lambda e^{-\lambda x} \lambda e^{-\lambda (t-x)}}
    {\lambda e^{-\lambda t} \frac{\lambda t}{1!}} = \frac{1}{t}
$$
因此在给定$X+Y=t$的前提下,随机变量$X$服从$(0,t)$上的均匀分布

\end{document}