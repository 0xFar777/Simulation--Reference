\documentclass{book}  
\usepackage{amsmath,lmodern} 
\usepackage{fancyhdr}
\usepackage{fontspec}
\usepackage[UTF8]{ctex}
\usepackage{ragged2e}
\usepackage{graphicx}
\usepackage{float}
\usepackage{listings}

\setmainfont{Times New Roman}
\renewcommand\tablename{Table}

\begin{document}
\zihao{4}
\centering
节选第四章 \\

\hspace*{\fill} \\
\zihao{5}
\justifying
\noindent
4.3.Give an efficient algorithm to simulate the value of a random variable X such that
\begin{center}
  $P\left\{X=1\right\} = 0.3$, \ \ \ $P\left\{X=2\right\} = 0.2$ \\
  $P\left\{X=3\right\} = 0.35$, \ \ \ $P\left\{X=4\right\} = 0.15$ \\
\end{center}
\noindent
翻译:给出一个有效的算法来模拟随机变量X的值,使得随机变量每个值的概率满足上式 \\
解:代码如下
\lstset{language = R}
\begin{lstlisting}
    #法一:sample抽样函数
    n <- 500000
    r <- sample(1:4, n, replace = TRUE, 
                prob = c(0.3,0.2,0.35,0.15))
    for(i in 1:4) {
        print(sum(r==i)/n)
    }
    #法二:逆变换法
    n <- 100000  
    x <- seq(1, 4, 1) 
    p <- c(0.3, 0.2, 0.35, 0.15)
    F <- c(0,cumsum(p))  #累加概率得到分布函数
    result <- vector(length = n)  
    r <- runif(n, min = 0, max = 1)
    for(i in 1:4){  
        T <- r > F[i] & r <= F[i+1]  
        #大于F[i]且不大于F[i+1]的就为TRUE,否则为FALSE
        #print(T) #放开注释以便理解过程
        result[T] <- x[i] 
        #上一步为FALSE的此次循环不赋值,等后面的循环
        #print(result)  #放开注释以便理解过程
    }
    table(result)/n  
\end{lstlisting}
第3题(法一)运行截图
\begin{figure}[H]
  \centering
  \includegraphics*[height = 3.8cm, width = 7.2cm]{gramFile/第三题(法一)运行截图.PNG}
  \caption{第三题(法一)R代码}
\end{figure}
第3题(法二)运行截图
\begin{figure}[H]
  \centering
  \includegraphics*[height = 6.4cm, width = 8.2cm]{gramFile/第三题(法二)运行截图.PNG}
  \caption{第三题(法二)R代码}
\end{figure}

\hspace*{\fill} \\

\noindent
4.7. A pair of fair dice are to be continually rolled until all the possible outcomes 2,3,...12
have occurred at least once. Develop a simulation study to estimate the expected number of dice rolls
that are needed. \\
\noindent
翻译:一对质量均匀的骰子不断投掷,直到所有可能的结果2,3,...,12至少出现一次.通过随机模拟研究估计预期的掷骰次数 \\
解:代码如下
\lstset{language = R}
\begin{lstlisting}
    n <- 6666
    count <- vector(length = n)
    for(i in 1:n) {
        k <- 2:12
        while(sum(k==0) < length(k)) {
            r1 <- sample(1:6, 1)
            r2 <- sample(1:6, 1)
            k[r1+r2-1] = 0
            count[i] = count[i] + 1
        }
    }
    mean(count) 
\end{lstlisting}
第7题运行截图
\begin{figure}[H]
  \centering
  \includegraphics*[height = 4.2cm, width = 6.2cm]{gramFile/第七题运行截图.PNG}
  \caption{第七题R代码}
\end{figure}
\noindent
结论:平均需要61次,才可以保证2-12每种结果都至少出现一次.

\hspace*{\fill} \\

\noindent
4.13. Let $X$ be a binomal random variable with parameters $n$ and $p$. Suppose that we
want to generate a random variable $Y$ whose probability mass function is the same
as the conditional mass function of $X$ given that $X \geq k$, for some $k \leq n$. Let
$\alpha = P\left\{X \geq k\right\}$ and suppose that the value of $\alpha$ has been computed. \\
(a) Give the inverse transform method for generating $Y$ \\
(b) Give a second method for generating $Y$ \\
(c) For what values of $\alpha$, small or large, would the algorithm in (b) be inefficient? \\
\noindent
翻译:假设$X$是一个具有参数$n$和$p$的二项式随机变量.现在我们想要生成一个随机变量$Y$,其概率质量函数与给定
$X \geq k$的$X$的条件质量函数相同,对于某些$k \leq n$. 假设$\alpha = P\left\{X \geq k\right\}$的值
已经计算出来. \\
(a)给出生成Y的逆变换方法 \\
(b)给出第二种生成Y的方法 \\
(c)对于$\alpha$,无论是小值还是大值,(b)中的算法是否效率低下? \\
解:(a)题代码如下 \\
\lstset{language = R}
\begin{lstlisting}
    f <- function(n, k, p) {
        num <- 100000
        x <- seq(k, n, 1) 
        prob_k <- vector(length = n-k+1)
        r <- rbinom(num, size = n, prob = p)
        alpha <- sum(r >= k)/num #P{X>=k}
        for(i in 1:length(prob_k)) {
          prob_k[i] <- sum(r == k+i-1)/num/alpha
          #P{X=k}/P{X>=k}
        }
        FF <- c(0,cumsum(prob_k))
        result <- vector(length = num) 
        R <- runif(num, min = 0, max = 1)
        for(i in 1:n-k+1){  
          T <- R > FF[i] & R <= FF[i+1]  
          result[T] <- x[i]
        }
        print(table(result)/num)
      }
      f(12,5,0.4)
\end{lstlisting}
第13题(a)运行截图
\begin{figure}[H]
  \centering
  \includegraphics*[height = 7.2cm, width = 10.2cm]{gramFile/第十三题(a)运行截图.PNG}
  \caption{第十三题(a)R代码}
\end{figure}
\noindent
解:(b)题代码如下
\lstset{language = R}
\begin{lstlisting}
    g <- function(n, k, p) {
        num <- 100000
        prob_k <- vector(length = n-k+1)
        r <- rbinom(num, size = n, prob = p)
        alpha <- sum(r >= k)/num  #P{X>=k}
        for(i in 1:length(prob_k)) {
          prob_k[i] <- sum(r == k+i-1)/num/alpha
          #P{X=k}/P{X>=k}
        }
        R <- sample(k:n, num, replace = TRUE,
                    prob = prob_k)
        print(table(R)/num)
      }
      g(12,5,0.4)
\end{lstlisting}
第13题(b)运行截图
\begin{figure}[H]
  \centering
  \includegraphics*[height = 5.2cm, width = 10.2cm]{gramFile/第十三题(b)运行截图.PNG}
  \caption{第十三题(b)R代码}
\end{figure}
\noindent
解:(c)题代码如下
\lstset{language = R}
\begin{lstlisting}
  #必须设置固定随机数,否则没有对比价值
  set.seed(100)  

  #(a)算法
  f <- function(n, k, p) {
    num <- 20222022  
    #必须保证num(实验次数)足够大,否则结论不太准确
    x <- seq(k, n, 1) 
    prob_k <- vector(length = n-k+1)
    r <- rbinom(num, size = n, prob = p)
    alpha <- sum(r >= k)/num #P{X>=k}
    print(alpha)
    for(i in 1:length(prob_k)) {
      prob_k[i] <- sum(r == k+i-1)/num/alpha
      #P{X=k}/P{X>=k}
    }
    FF <- c(0,cumsum(prob_k))
    result <- vector(length = num) 
    R <- runif(num, min = 0, max = 1)
    for(i in 1:n-k+1){  
      T <- R>FF[i] & R<= FF[i+1]  
      result[T] <- x[i]
    }
  }

  #(b)算法
  g <- function(n, k, p) {
      num <- 20222022
      prob_k <- vector(length = n-k+1)
      r <- rbinom(num, size = n, prob = p)
      alpha <- sum(r >= k)/num  #P{X>=k}
      for(i in 1:length(prob_k)) {
        prob_k[i] <- sum(r == k+i-1)/num/alpha
        #P{X=k}/P{X>=k}
      }
      R <- sample(k:n, num, replace = TRUE,
                  prob = prob_k)
    }

  #改变k的值相当于改变alpha的值
  aTime <- vector(length = 6)
  bTime <- vector(length = 6)
  for(i in 1:6) {
    ptmf <- proc.time()
    f(12,i,0.4)
    aTime[i] <- (proc.time() - ptmf)[3]
    print(proc.time() - ptmf)
    ptmg <- proc.time()
    g(12,i,0.4)
    bTime[i] <- (proc.time() - ptmg)[3]
    print(proc.time() - ptmg)
  }

  #以下绘制(a)和(b)两种算法运行时间对比折线图
  library(ggplot2)
  dataR <- data.frame(
    number = c(seq(1,6,1),seq(1,6,1)),
    data = c(aTime, bTime), 
    belong = c(rep("a",6), rep("b",6))
    )
  p <- ggplot(data = dataR,
    mapping = aes(
      x = number,
      y = data,
      ))
  p + geom_line(mapping = aes(color = belong)) +
    geom_point(
      size = 2,
      shape = 21,
      fill = "purple"
      ) + 
    labs(
      title = "(a)和(b)两种算法运行时间对比",
      x = "alpha的值(越往右越小)",
      y = "运行时间"
    ) +
    theme(plot.title = element_text(color = "blue",
                                    face = "bold",
                                    hjust = 0.5
                                    ),
          axis.text.x = element_text(
            size = 10, color = "black"
            ),
          axis.text.y = element_text(
            size = 10, color = "black"
            ),
          axis.title.x = element_text(
            size = 10, color = "black"
            ),
          axis.title.y = element_text(
            size = 10, color = "black"
            )
          )
\end{lstlisting}
第13题(c)运行截图
------(a)和(b)两种算法运行数据
\begin{figure}[H]
  \centering
  \includegraphics*[height = 13.14cm, width = 3.14cm]{gramFile/第十三题(c)两种算法运行数据.PNG}
  \caption{第十三题(c)两种算法运行数据}
\end{figure}
------(a)和(b)两种算法运行时间对比
\begin{figure}[H]
  \centering
  \includegraphics*[height = 5.8cm, width = 10.2cm]{gramFile/第十三题(c)两种算法运行时间对比折线图.PNG}
  \caption{第十三题(c)两种算法运行时间对比}
\end{figure}
\noindent
结论:对比(a)和(b)两种算法,发现无论$\alpha$的取值是大是小,(a)算法的运行时间总是大于(b)算法,
因此(b)中的算法效率并不低下,其运行效率要高于(a)算法;
而在算法内(无论对于(a)还是(b)算法),随着$\alpha$取值的减小,运行时间也有所减少 \\
注意:要确保在程序运行过程中电脑没有其他耗费CPU的程序在运行,否则会严重影响到测算结果

\hspace*{\fill} \\

\noindent
4.16.Present a method to generate the value of $X$, where \\
$$
  P\left\{X=j\right\} = (\frac{1}{2})^{j+1} + \frac{(\frac{1}{2})2^{j-1}}{3^{j}},
  \ \ \ j = 1,2,...
$$
Use the text's random number sequence to generate $X$. \\
\noindent
翻译:提出一种生成$X$值的方法,使用随机模拟的方法生成满足上式中每个取值概率的随机变量  \\
解:代码如下
\lstset{language = R}
\begin{lstlisting}
    num <- 100000
    prob_N <- vector(length = num)
    count <- 0;
    for(j in 1:length(prob_N)) {
      prob_N[j] <- 0.5^(j+1) + 0.5*2^(j-1)/3^j
      if(sum(prob_N[1:j]) >= 1) {
        break
      }
      count <- count + 1
    }
    R <- sample(1:count, num, replace = TRUE,
                prob = prob_N[1:count])
    print(table(R)/num)
\end{lstlisting}
第16题运行截图
\begin{figure}[H]
  \centering
  \includegraphics*[height = 6cm, width = 12.2cm]{gramFile/第十六题运行截图.PNG}
  \caption{第十六题R代码}
\end{figure}

\hspace*{\fill} \\

\noindent
4.19.A random selection of $m$ balls is to be made from an urn that contains $n$ balls,
$n_{i}$ of which have color type $i$, $\Sigma_{i=1}^{r} n_{i} = n$. Discuss efficient
procedures for simulating $X_{1}$,..., $X_{r}$, where $X_{i}$ denotes the number of
withdrawn balls that have color type $i$. \\
翻译:从包含$n$个球的箱中随机抽取$m$个球,\ $n_{i}$代表颜色种类为$i$的球的个数,且$\Sigma_{i=1}^{r}
  n_{i} = n$,探究如何设计一个有效的程序模拟$X_{1}$,...,$X_{r}$,其中$X_{i}$代表颜色种类为$i$
抽取的个数 \\
解:代码如下
\lstset{language = R}
\begin{lstlisting}
    #n是箱中球的总个数,m是抽取的球的数量,r是球的颜色种类
    f <- function(n, m, r) {
        p <- vector(length = r)
        #随机生成每种颜色的球的概率
        R <- runif(r-1, min = 0, max = 1)
        R <- sort(R)
        for(i in 1:length(p)) {
          if(i == 1) {
            p[i] <- R[i]
          } else if(i == length(p)) {
            p[i] <- 1-R[i-1]
          } else {
            p[i] <- R[i] - R[i-1]
          }
        }
        print(round(p,3))
        #根据每种球的概率模拟生成箱中的n个球
        all <- sample(1:r, n, replace = TRUE, prob = p)
        #从包含n个球的箱中随机抽取m个(不放回)
        result <- sample(all, m, replace = FALSE)
        #输出m个球中不同颜色球的概率
        print(round(table(result)/m,3))
      }
\end{lstlisting}
第19题运行截图
\begin{figure}[H]
  \centering
  \includegraphics*[height = 4cm, width = 5cm]{gramFile/第十九题运行截图.PNG}
  \caption{第十九题R代码}
\end{figure}
\noindent
结论:由输出结果可知:总体数量不变时,样本量越大,抽取到的每种颜色的球的概率就越接近总体中
每种颜色的球的概率 \\

\end{document}